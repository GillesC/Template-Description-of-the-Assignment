% created by Gilles Callebaut
% Date: 04/11/2022
% URL: https://github.com/GillesC/Template-Description-of-the-Assignment

\documentclass{article}
\usepackage[utf8]{inputenc}
\usepackage{geometry}
 \geometry{
 a4paper,
 total={170mm,257mm},
 left=20mm,
 top=20mm,
 }
 \usepackage{graphicx}
 \usepackage{titling}
 \usepackage{lipsum}  
 \usepackage{cmbright}
 \usepackage{iflang}
\usepackage[table]{xcolor}


 \usepackage{pdflscape} % for lanscape Gantt chart
 \usepackage{pgfgantt}
 \definecolor{ColorGent}{RGB}{78, 169, 191}
\definecolor{ColorLeuven}{RGB}{153,	153,255}
\definecolor{ColorCommon}{RGB}{50,50,50}
\ganttset{group/.append style={ColorCommon},
milestone/.append style={red},
progress label node anchor/.append style={text=red}}


\newcommand{\instruction}[1]{{\color{blue}{\textbf{Instructions}: {#1}} }}


 %%%%%%%%%%%%% define header/footer %%%%%%%%%%%%%
 \usepackage{fancyhdr}
\fancypagestyle{plain}{%  the preset of fancyhdr 
    \fancyhf{} % clear all header and footer fields
    \fancyfoot[R]{\includegraphics[width=2cm]{KULEUVEN_GENT_RGB_LOGO.png}}
    \fancyfoot[L]{\thedate}
    \fancyhead[L]{\iflanguage{english}{Description of Assignment}{Opdrachtomschrijving}}
    \fancyhead[R]{\theauthor}
}
%%%%%%%%%%%%%%%%%%%%%%%%%%%%%%%%%%%%%%%%%%%%%%%%%

\setlength{\headsep}{0pt} % removes redudant vspace between document and header

%%%%%%%%%%%%% update \maketitle to show title + student/promotors %%%%%%%%%%%%%
\makeatletter
\def\@maketitle{%
  \newpage
  \null
  \vskip 1em%
  \begin{center}%
  \let \footnote \thanks
    {\LARGE \@title \par}%
    \vskip 1em%
    %{\large \@date}%
  \end{center}%
  \par
  \vskip 1em
    \noindent\begin{tabular}{@{}ll}
    Student & \theauthor\ (\class)\\
     Promotor &  \promotor\\
     Co-promotors & \copromotor
    \end{tabular}
    }
\makeatother
%%%%%%%%%%%%%%%%%%%%%%%%%%%%%%%%%%%%%%%%%%%%%%%%%%%%%%%%%%%%%%%%%%%%%%%%%%%%%%%


%%%%%%%%% UPDATE FOLLOWING ITEM %%%%%%%%%%%
\title{Advances in Single and Multi-Antenna Technologies for Energy-Efficient IoT}
\author{Gilles Callebaut}
\date{November 2022}
\def\promotor{dr. Gilles Callebaut}
\def\copromotor{ing. Jarne Van Mulders, ing. Guus Leenders}
\def\class{MELICTes}
\usepackage[english]{babel} % change english or dutch
%%%%%%%%%%%%%%%%%%%%%%%%%%%%%%%%%%%%%%%%%%

\begin{document}

\maketitle

% uncomment below
\vspace{1em}
\instruction{
 This template is without obligation. Feel free to change this template however you like. The description of assignment should, however, include the following:  
 title, names of the students, promotor and co-promotors, date, sections: positioning of the thesis, objectives and planning.
    Use prefixes or postfixes to indicate the version or date. In this way, both you and your supervisors have the same naming convention, e.g., \textit{description\_assigment\_20221104.pdf}.
    Discuss this description with your supervisors. Note: you may have other tasks deadline and for sure your supervisor(s) do. Hence, plan!
}


\section*{\iflanguage{english}{Positioning of the thesis}{Situering van het eindwerk}}
\instruction{Sketch the context and the problem statement in an appealing manner, such that it can also attract interest from non-experts.}

%\lipsum[1-1]

\section*{\iflanguage{english}{Objectives}{Doelstellingen}}
\instruction{You formulate a few very concrete objectives you want to realise in your thesis. This can be a list-format.
The objectives are the initial research questions. These objectives serve as a basis to evaluate the results obtained at the end of your thesis.}

%\lipsum[3-3]

\section*{Planning}
\instruction{Make a work plan. Divide your work in work packages and define a few key milestones.
Take into account dependencies between tasks.}

\clearpage
\begin{landscape}
    \centering
    %!TEX root = main.tex

\ganttset{group/.append style={orange},
MileRed/.style={milestone/.append style={fill=red}},
MileGreen/.style={milestone/.append style={fill=green, shape=circle}},
progress label node anchor/.append style={text=red}}

     \begin{ganttchart}[%Specs
     y unit title=0.5cm,
     y unit chart=0.7cm,
     x unit=0.4cm,
     vgrid,hgrid,
     title height=1,
%     title/.style={fill=none},
     title label font=\bfseries\footnotesize,
     bar/.style={fill=ColorGent},
     bar height=0.7,
%   progress label text={},
     group right shift=0,
     group top shift=0.7,
     group height=.3,
     group peaks width={0.2},
     inline]{1}{48}
    %labels
    \gantttitle[]{Year 1}{12}                 % title 1
    \gantttitle[]{Year 2}{12}
    \gantttitle[]{Year 3}{12}
    \gantttitle[]{Year 4}{12} \\
    \gantttitle{Q1}{3}                      % title 2
    \gantttitle{Q2}{3}
    \gantttitle{Q3}{3}
    \gantttitle{Q4}{3}
    \gantttitle{Q1}{3}
    \gantttitle{Q2}{3}
    \gantttitle{Q3}{3}
    \gantttitle{Q4}{3}
    \gantttitle{Q1}{3}                      % title 2
    \gantttitle{Q2}{3}
    \gantttitle{Q3}{3}
    \gantttitle{Q4}{3}
    \gantttitle{Q1}{3}
    \gantttitle{Q2}{3}
    \gantttitle{Q3}{3}
    \gantttitle{Q4}{3}\\
    % Setting group if any
    %\ganttgroup[inline=false]{WP 1}{1}{48}\\
    \ganttbar[inline=false]{T1.1: State of the Art}{1}{48}\\
    \ganttbar[inline=false]{T1.2: Framework simulation}{1}{3}
    \ganttbar[inline=false]{}{16}{18}\\
    \ganttmilestone[inline=false,MileRed]{MS1: simulation model for MTC}{3} \\
    \\

    %\ganttgroup[inline=false]{WP 2}{3}{30} \\
    \ganttbar[inline=false]{T2.1: Cross-layer solution NB-IoT}{4}{9} \\
    \ganttbar[inline=false]{T2.2: Ultra low power MTC}{19}{31} \\
    \ganttmilestone[inline=false,MileRed]{MS3: Cross-layer design}{31} \\
    \\

    %\ganttgroup[inline=false]{WP 3}{3}{42} \\
    \ganttbar[inline=false]{T3.1: Test-bed development}{10}{12}
    \ganttbar[inline=false]{}{32}{35} \\
    \ganttbar[inline=false]{T3.2: Experimental validation}{13}{15}
    \ganttbar[inline=false]{}{36}{42} \\
    \ganttmilestone[inline=false,MileRed]{MS2: NB-IoT transmission validation}{15} \\
    \ganttmilestone[inline=false,MileRed]{MS4: Ultra low power MTC validation}{42} \\
    \\
    \ganttbar[inline=false]{T4.1: Dissemination}{9}{48} \\
    \ganttbar[inline=false]{T4.2: Report}{23}{24}
    \ganttbar[inline=false]{}{43}{48} \\
    \ganttmilestone[inline=false,MileRed]{MS5: PhD thesis}{48} \\

    % PAPERS
    \ganttmilestone[inline=false,MileGreen]{Papers}{9}
    \ganttmilestone[inline=false,MileGreen]{}{18}
    \ganttmilestone[inline=false,MileGreen]{}{30}
    \ganttmilestone[inline=false,MileGreen]{}{42}
    \ganttmilestone[inline=false,MileGreen]{}{48}\\
\end{ganttchart}

\end{landscape}

\end{document}
